\part*{Appendix}
\addcontentsline{toc}{chapter}{Appendix}
\renewcommand{\thesection}{\arabic{section}}
\renewcommand{\theequation}{\arabic{equation}}
\section{Appendix One: Consumer Application}\label{section:Appendix One: Consumer Application}

The Interdisciplinary Project consists of two parts. The first one is the scientific paper, which was described in the pages before. The second part is a consumer application which will be described in this part.\newline
The aim of the application is to give the person, who just placed an order, detailed  information about the delivery. This includes the current step in the delivery process as well as the current position of the driver on a map. After the process the customer will be able to rate the delivery.\newline
The moment the customer finishes placing an order the backend will create an account for the customer. The credentials for the new user will be sent to the customer by email. The email also directs to the application in the Play Store where it can be downloaded. This part has to be implemented in the backend if it would be used in a real scenario.\newline
The user can login into the application (Fig. \ref{fig:consumer_application}, left) by entering the user data. Right now a driver account is used to demonstrate the functionality, since the backend does not support user right manage in the database. There is no user in the database except the operator and the driver who can see the order status.\newline
When the customer enters the application, the screen in the middle of Figure \ref{fig:consumer_application} is shown. In this screen \texttt{Order is getting assigned} is shown as status. This means the driver cannot be tracked yet since the order has not been assigned to a driver. Since the delivery time is yet unknown, it is set to one hour. This happens because the algorithm on the server does not assign the driver instantly. The driver will only be informed when he should head to the restaurant to arrive on time when the food is ready.\newline
The application queries every 10 seconds for a status change on the server. As soon as the driver has accepted the order the status switches to \texttt{Driver is on the way to the Restaurant} (Figure \ref{fig:consumer_application}, right). From now on the position of the driver is updated every 10 seconds so the customer using the application always sees where her delivery is at the moment. The customer sees the different steps of the delivery process, namely \texttt{pickup\_ended}, when the driver has entered the restaurant, and \texttt{delivery\_started}, when the driver has left the restaurant with the meal (Figure \ref{fig:consumer_application2}, left and middle). The information for the estimated time of delivery is taken from the server. It consists of the predicted preparation time, which was calculated in the scientific paper and the driving time Google Maps suggests for the route. A buffer is added to increase the experience for the customer since the driver will arrive early, if nothing happens, and has some buffer in case there is a traffic jam.
\begin{figure}[htp]

\centering
\includegraphics[width=.3\textwidth]{images/1_login.png}\hfill
\includegraphics[width=.3\textwidth]{images/6_assign.png}\hfill
\includegraphics[width=.3\textwidth]{images/2_pickup_started.png}
\caption{The consumer application. Login on the left. Driver has to be assigned in the middle. Delivery pick up has started on the right.}
\label{fig:consumer_application}

\end{figure}

\begin{figure}[htp]

\centering
\includegraphics[width=.3\textwidth]{images/3_pickup_ended.png}\hfill
\includegraphics[width=.3\textwidth]{images/4_delivery_started.png}\hfill
\includegraphics[width=.3\textwidth]{images/5_feedback.png}
\caption{The graphical user interface of the consumer application. When the driver is at the restaurant (left), when the driver has left the restaurant (middle) and when the driver has finished the delivery (right).}
\label{fig:consumer_application2}

\end{figure}

As soon as the driver has completed the delivery by giving the order to the customer and checking it in her driver application, a feedback dialog opens in the customer application (Figure \ref{fig:consumer_application2}, right). In this popup the customer can rate her experience and provide valuable feedback for VOLO in case something was not as she wished.\newline
After entering the feedback, the application is closed and the user is logged out from the application. The user account is now set to inactive on the server since there is nothing to track anymore. In case the user orders again, the account will be reactivated and a new password will be generated. The user receives an email with her new login credentials.\newline
This way the customer tracking application provides an easy and comfortable way of tracking an order and giving feedback about the delivery.
\newpage
\section{Appendix Two: Code and Results}\label{section:Appendix Two: Code and Results}

The code and application will be send by email since they contain confidential material of VOLO UG.
